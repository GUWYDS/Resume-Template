% 模板基础配置(文档类与核心样式)
\documentclass[11pt,a4paper,roman]{moderncv}  % 文档类设置:字体大小11pt,A4纸,罗马字体
\moderncvstyle{banking}                       % 简历样式:banking(银行式布局)
\moderncvcolor{blue}                          % 主题颜色:蓝色

% 编码与包引入
\usepackage[utf8]{inputenc}                   % 字符编码:UTF-8
\usepackage{fontawesome}                      % 引入字体图标(用于联系方式等)
\usepackage{tabularx}                         % 增强表格功能
\usepackage{ragged2e}                         % 文本对齐控制
\usepackage[scale=0.8]{geometry}              % 页面布局缩放:0.8倍
\usepackage{multicol}                         % 多列布局支持
\usepackage{import}                           % 外部文件导入功能

% 页面设置
\nopagenumbers{}                              % 不显示页码

% 个人信息配置
\name{Yifei}{Wang}                            % 姓名

% 自定义命令(用于内容格式化)
% 1. 自定义经历条目(包含职位、时间、地点等信息)
\newcommand*{\customcventry}[7][.25em]{
  \begin{tabular}{@{}l} 
    {\bfseries #4}
  \end{tabular}
  \hfill
  \begin{tabular}{l@{}}
     {\bfseries #5}
  \end{tabular} \\
  \begin{tabular}{@{}l} 
    {\itshape #3}
  \end{tabular}
  \hfill
  \begin{tabular}{l@{}}
     {\itshape #2}
  \end{tabular}
  \ifx&#7&%
  \else{\\%
    \begin{minipage}{\maincolumnwidth}%
      \small#7%
    \end{minipage}}\fi%
  \par\addvspace{#1}}

% 2. 自定义项目条目(用于项目、奖项等)
\newcommand*{\customcvproject}[4][.25em]{
  \begin{tabular}{@{}l} 
    {\bfseries #2}
  \end{tabular}
  \hfill
  \begin{tabular}{l@{}}
     {\itshape #3}
  \end{tabular}
  \ifx&#4
  \else{\\%
    \begin{minipage}{\maincolumnwidth}%
      \small#4%
    \end{minipage}}\fi%
  \par\addvspace{#1}}

% 3. 自定义参考条目
\newcommand*{\cvref}[3][.25em]{
  \begin{tabular}{@{}l} 
    {\bfseries #2}
  \end{tabular}
  \ifx&#3
  \else{\\%
    \begin{minipage}{\maincolumnwidth}%
      \small#3%
    \end{minipage}}\fi%
  \par\addvspace{#1}}

% 表格间距调整
\setlength{\tabcolsep}{12pt}

% 文档内容开始
\begin{document}

% 生成标题并调整间距
\makecvtitle
\vspace*{-23mm}  % 缩减标题下方空白
\vspace{1em}     % 增加标题与内容的间距

% 联系方式(使用图标与表格布局)
\begin{center}
\begin{tabular}{ c c c c }
 \faLink\enspace sumanrbt1997.wixsite.com/suman & \faEnvelopeO\enspace sumanrbt1997@gmail.com & \faGithub\enspace github.com/suman1209 
 \\\faMobile\enspace +94 77 264 6761 & \faLinkedin\enspace in/sumanrbt97
\end{tabular}
\end{center}

% 职业目标 section
\section{CAREER OBJECTIVE}
\begin{itemize}
  \item I am very much interested and looking forward to involve in \textbf{Real World Projects}.
  \item I want to work with a team where I can apply my knowledge and gain practical and professional experience.
  \item \textbf{Currently seeking internship in the field of my interest} – Robotics and AI, Machine learning, Computer Vision, Embedded systems, Processor Design.
\end{itemize}

% 教育背景 section
\section{EDUCATION}
\customcventry{Oct 2016 -- Present}{BSc. Eng (Hons.)}{University of Moratuwa}{Moratuwa, Srilanka}{}{Electronics and Telecommunication Engineering}
\begin{itemize}
  \item Cumulative GPA -- \textbf{3.45/4.2}
\end{itemize}

\customcventry{May 2013 -- Sep 2015}{G.C.E Advanced Level Examination}{Jaffna Hindu College}{Jaffna, SriLanka}{}{}
\begin{itemize}
  \item \textbf{3A's} (Combined Mathematics, Physics, Chemistry).
\end{itemize}

\customcventry{Jan 2010 -- Dec 2012}{G.C.E Ordinary Level Examination}{Manipay Angel International School}{Manipay, SriLanka}{}{}
\begin{itemize}
  \item \textbf{6A's} (Including Information Communication Technology, German language, English Literature).
\end{itemize}

% 技术技能 section
\section{ADVANCED TECHNICAL SKILLS \& KNOWLEDGE AREAS}
\begin{itemize}
  \item Programming Language --\textbf{Python} [good] ,\textbf{C\#,C++} [Basic]
  \item \textbf{MATLAB}
  \item \textbf{SolidWorks} [good]
  \item \textbf{OpenCV python}, [in Anaconda and Jupyter Lab]
\end{itemize}

% 专业资质 section
\section{PROFESSIONAL QUALIFICATION}
\begin{itemize}
  \item \textbf{Data structures and Algorithm}
  \item \textbf{PCB designing} in Altium and making complete circuit
  \item \textbf{Digital circuit design} (tools – DE 2-115 FPGA,Verilog , Quartus , Modelsim Altera)
  \item \textbf{Knowledge of processor architecture}
  \item Fundamentals of Image processing
  \item Familiarity with microcontrollers.
\end{itemize}

% 语言能力 section
\section{LANGUAGE}
\begin{itemize}
  \item {\textbf{English} [fluent]}
  \item {\textbf{Deutsch} [good]}
  \item {\textbf{Tamil} [fluent]} 
\end{itemize}

% 奖项与竞赛 section
\section{AWARDS AND COMPETITIONS}
% 将列表作为第四个参数传入,用 {} 包裹
\customcvproject{Intra University Robotics Competition}{2017,2018}{
  \begin{itemize}
    \item Built a four wheeled robot for a given task
  \end{itemize}
}

% 其他奖项也按此格式修改
\customcvproject{Asian Finalists in IEEE SS12}{2018}{
  \begin{itemize}
    \item Innovation Challenge
  \end{itemize}
}

\customcvproject{Finalists in Yarl Geek Challenge}{2018}{
  \begin{itemize}
    \item Start-up Competition
  \end{itemize}
}

\customcvproject{IEEE Electronic Design and Competition}{2018}{
  \begin{itemize}
    \item Built a DC to DC Voltage regulator 
  \end{itemize}
}

% 项目经历 section
\section{PROJECTS}
\textbf{For detailed description of all my projects, refer the link below.} \\
\faLink \, https://sumanrbt1997.wixsite.com/suman

\customcvproject{Designing and implementing a processor for down sampling an image}{Ongoing}{
    \begin{itemize}
      \item FPGA [DE2-115], Quartus prime lite, Verilog
    \end{itemize}
}

\customcvproject{UART Transceiver on FPGA}{2018}{
    \begin{itemize}
      \item Verilog, FPGA
      \item Programming and testing of UART communication protocol on FPGA [DE2-115].
    \end{itemize}
}

\customcvproject{Created a four wheeled Robot for a given Task}{2018}{
    \begin{itemize}
      \item Arduino microcontroller
    \end{itemize}
}


% 领导经历 section
\section{LEADERSHIP ACTIVITIES}
\begin{itemize}
  \item \textbf{Led} the team which conducted Education Ministry Sponsored \textbf{Robotics Workshop} in the \textbf{Northern Province} (2018)
  \item \textbf{Department representative} for semesters 4 and 5 [Jun 2018 – Jun 2019]
  \item \textbf{Taught CIE A/L physics and been the Lab exam supervisor for CIE (2016) \\ Centre} – Angel International School Manipay \textbf{[Aug 2015 – Aug 2016]}
\end{itemize}

% 布局参数调整(针对多列布局)
\setlength{\columnwidth}{\textwidth}
\addtolength{\columnwidth}{-\columnsep}
\setlength{\columnwidth}{.5\columnwidth}
\setlength{\hintscolumnwidth}{0.175\columnwidth}
\setlength{\separatorcolumnwidth}{0.025\columnwidth}

\renewcommand*{\recomputecvbodylengths}{%
  \setlength{\maincolumnwidth}{\columnwidth-\leftskip-\rightskip-\separatorcolumnwidth-\hintscolumnwidth}%
  \setlength{\listitemcolumnwidth}{\maincolumnwidth-\listitemsymbolwidth}%
  \setlength{\doubleitemcolumnwidth}{\maincolumnwidth-\hintscolumnwidth-\separatorcolumnwidth-\separatorcolumnwidth}%
  \setlength{\doubleitemcolumnwidth}{0.5\doubleitemcolumnwidth}%
  \setlength{\listdoubleitemcolumnwidth}{\maincolumnwidth-\listitemsymbolwidth-\separatorcolumnwidth-\listitemsymbolwidth}%
  \setlength{\listdoubleitemcolumnwidth}{0.5\listdoubleitemcolumnwidth}%
  \setlength{\parskip}{0pt}}
\recomputecvbodylengths

\end{document}